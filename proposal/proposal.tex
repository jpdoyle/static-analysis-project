\documentclass[twocolumn]{article}

\usepackage{hyperref}

\title{17-355 Research Project Proposal: Detecting and Flagging Self-Timing
Code}
\author{Cam Wong, Joe Doyle}
\begin{document}
\maketitle

\section{Motivation}

``Timing attacks" are a broad set of attacks that can be used to
covertly transmit information or inspect the internal workings of
otherwise-isolated programs. Modern microprocessors have many complex
execution strategies for their instruction sets, but they are designed
so that every execution path is semantically indistinguishable (at
least, in theory). However, although the semantics stay the same,
performance characteristics -- primarily how much time a sequence of
instructions takes to complete -- can vary depending on the state of
the processor and the particular execution strategy used. When
performance-critical resources (such as caches) are shared, processes
can influence these resources to signal across VM sandboxes, or they
can watch cache effects to try to extract security-critical data, or
(in the case of Spectre) they can take advantage of processor flaws to
inspect kernel memory.

There is existing work doing dynamic analysis of programs to detect
attempts to certain kinds of attacks in progress, and there are
analysis tools and techniques to make cryptography software less
vulnerable to this kind of attack, but to our knowledge there is no
existing framework for statically analyzing whether a piece of
software attempts to invoke a timing-based attack.

\section{Project Description}

Our goals for this project are to provide, with reasonable confidence,
a list of ``critical sections'' that are possibly being measured via
an ``internal timer''. We intend to do this by examining calls to (and
stores of calls to) ``timer functions'' (a list of which can be passed
into the analysis as an input).

For a proof of concept, we will be performing the analysis on C code,
to leverage the more mature parser libraries. A ``real world''
implementation of this analysis might instead work with Javascript or
x86. We will also only be focusing on single procedure analyses, but
may consider extending to full-program analysis after the checkpoint
should time permit.

For the checkpoint, we aim to have some scaffolding that will demarcate
all calls to timer functions and where their results are used. From there,
we can extend the analysis to detect whether those results are compared
in some way (such as subtracting them), and use that to determine critical
sections.

\section{Papers}

We'd like to approach this as a research variant of the project. As
such, we've identified several papers that could be relevant. Most of
these papers are on arXiv, rather than coming from journals or
conferences -- we had a difficult time finding relevant papers in
journals, so if any of the instructors have suggestions for relevant
conference papers, we'd be happy to have them.

The blog post on Spectre:
\url{https://googleprojectzero.blogspot.com/2018/01/reading-privileged-memory-with-side.html},
and the paper: \url{https://spectreattack.com/spectre.pdf}.

``MeltdownPrime and SpectrePrime: Automatically-Synthesized Attacks
Exploiting Invalidation-Based Coherence Protocols":
\url{https://arxiv.org/abs/1802.03802}

``The Spy in the Sandbox -- Practical Cache Attacks in Javascript'':
\url{The Spy in the Sandbox -- Practical Cache Attacks in Javascript}

``Systematic Classification of Side-Channel Attacks: A Case Study for
Mobile Devices'': \url{https://arxiv.org/abs/1611.03748}

``Robust and Efficient Elimination of Cache and Timing Side
Channels'': \url{https://arxiv.org/abs/1506.00189}

``Rigorous Analysis of Software Countermeasures against Cache Attacks'': \url{https://arxiv.org/abs/1603.02187}

\end{document}

