\documentclass[twocolumn]{article}
\title{17-355 Research Project Proposal: Detecting and Flagging Self-Timing
Code}
\author{Cam Wong, Joe Doyle}
\begin{document}
\maketitle

\section{Motivation}

``Timing attacks" are a broad set of attacks that can be used to
covertly transmit information or inspect the internal workings of
otherwise-isolated programs. Modern microprocessors have many complex
execution strategies for their instruction sets, but they are designed
so that every execution path is semantically indistinguishable (at
least, in theory). However, although the semantics stay the same,
performance characteristics -- primarily how much time a sequence of
instructions takes to complete -- can vary depending on the state of
the processor and the particular execution strategy used. When
performance-critical resources (such as caches) are shared, processes
can influence these resources to signal across VM sandboxes, or they
can watch cache effects to try to extract security-critical data, or
(in the case of Spectre) they can take advantage of processor flaws to
inspect kernel memory.

There is existing work doing dynamic analysis of programs to detect
attempts to certain kinds of attacks in progress, and there are
analysis tools and techniques to make cryptography software less
vulnerable to this kind of attack, but to our knowledge there is no
existing framework for statically analyzing whether a piece of
software attempts to invoke a timing-based attack.

\section{Project Description}

Our goals for this project are to provide, with reasonable confidence,
a list of ``critical sections'' that are possibly being measured via
an ``internal timer''. We intend to do this by looking for functions that
are monotonic with respect to time, then finding sections of code
``between'' calls to these functions.

For a proof of concept, we will be performing the analysis on C code,
to leverage the more mature parser and dataflow libraries. A ``real world''
implementation of this analysis might instead work with Javascript.

\section{Papers}



\end{document}

