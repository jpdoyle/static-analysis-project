
\documentclass{article}
\usepackage{multicol}

\title{Detecting and Flagging Self-Timing Code}
\author{Cameron Wong, Joe Doyle}

\begin{document}
\maketitle

\begin{abstract}
  With the advent of the SPECTRE vulnerability bringing new attention to
  hardware-level timing-based attacks, much research has gone into mitigating
  these at a software level. Most of these efforts approach the problem by
  disabling the attack vector entirely, but have so far shown to have serious
  performance ramifications. We implemented a source-level static analysis of
  C code that could detect and mark ``timed sections'' based on further
  analyses to recognize ``timer functions'', which might be extended to
  statically determine whether a given source is attempting to execute a
  timing-based attack.
\end{abstract}

\begin{multicols*}{2}
  \section{Background}

  ``Timing attacks'' are a broad category of attacks that can be used to
  covertly transmit information or inspect the internal state of a program by
  taking advantage of specific properties about the hardware. An important
  consequence about the physical nature of these attacks is that this can be
  used to bypass software-level protections by influencing hardware-internal
  processes and measuring the time taken to execute. The SPECTRE bug, for
  example, takes advantage of speculative execution and hardware caching
  effects to leak the contents of kernel memory.
\end{multicols*}

\end{document}

